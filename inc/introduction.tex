\chapter{Introduction}
\label{chap:introduction}

% quick background
Bitcoin is a peer-to-peer electronic payment system relying on cryptography and a distributed ledger known as the blockchain~\cite{nakamoto2008bitcoin}. 
The purpose of the blockchain is to record all transactions done in the system, and is maintained collectively by the network. New transactions is added to the blockchain by spending computational power; making the transactions practically irreversible as the blockchain grows and more computational power is used, thus removing the need for trusted third parties to verify transactions. Such a decentralized payment solution does however not scale very well. It can handle maximally around 7 transaction per second \cite{poon2015bitcoin}, compared to a centralized payment system like visa, which can handle up to 47,000 transactions per second \cite{visa_stress}.
A proposed solution to help scale the Bitcoin system is payment channel networks. Such a network allows users to do transactions without them being published in the Bitcoin network, and they are therefore not included in the blockchain, which is why they are known as off-chain transactions. This makes it possible for more transactions being done without increasing the capacity of the Bitcoin system itself. It can be thought of as a layer on top of Bitcoin, since it requires the Bitcoin system and builds on top of it. As the name implies, these networks consists of interconnected payment channels. A payment channel is a one-to-one channel allowing two participants to exchange founds between each other. A network is created by connecting multiple channels with different users, making it possible for transfers to be done across multiple channels-e.g., Alice can pay Bob without having a channel directly connecting them, but using intermediary channels. These off-chain transactions done inside this network does not impact the Bitcoin network, only the management of the payment channels does.
\\ 

%problem description
As the blockchain is a public authoritative record of transactions done within the Bitcoin system, it has been used in research projects focusing on anonymity and privacy of users using the system~\cite{reid2013analysis,meiklejohn2013fistful}. Because the data in the blockchain provide a complete record of how founds have moved, the focus of the research projects has been to analyze the data on the blockchain, and determine user involvement in transactions and founds. This is done by grouping and contextualizing the information there, and successfully doing so will provide one with a record of all user activity. 
The Lightning Network (LN) as described in the paper by Poon and Dryja~\cite{poon2015bitcoin}, is a payment channel network currently used with Bitcoin. The use of this second layer network allowing transactions to be done off-chain, makes this type of blockchain analysis more difficult. As the Lightning Network (LN) is built on top of Bitcoin, there is a need for on-chain transactions to manage the payment channels it consists of. These on-chain transactions from the LN will be recorded on the blockchain just as other Bitcoin transactions, and the data contained in them will be validated same as any other transactions; they will however not provide explicit record of how founds have moved inside the LN, unlike the Bitcoin transactions do for the Bitcoin network.
This means one can still analyze the blockchain to see how founds move inside the Bitcoin network, but one can not do this to the same degree for the LN. Jordi Herrera-Joancomartí and Cristina Pérez-Solà~\cite{herrera2015research} points this out by stating that the methods used for blockchain analysis in earlier research, is no longer effective if payment channel networks would be widely used. In the case of the LN being widely used, the majority of transactions would be done off-chain and the only on-chain transactions would be the ones required for the payment channels to operate. Giulio Malavolta et al.~\cite{malavolta2017concurrency} identifies some possible privacy implications of the on-chain transactions from payment channels networks, so there might be possibilities of using blockchain data related to payment channels networks to reveal user information. Now that there are functioning payment channel networks such as the Lightning Network, we can explore what information will be available from the on-chain transactions; not just in a theoretical manner, but also the impact of real user behaviour and large data sets. In this thesis we will explore the possibility of blockchain analysis in relation to the lightning network, and to what degree it can impact privacy of the users. Doing this while the technology is still in a developmental stage and not yet widely used can be important, as resolving potential privacy risks early is desirable.
\\

This project will explore the implications of LN from the blockchain perspective.
We will attempt to answer our research question which is: what information from the LN is available in the Bitcoin system, and how does it impact the privacy of the users?
Our main goal, which will allow us to answer this, is to create a method for identifying transactions related to the lightning network in the blockchain. By achieving this we will discover what information is available to us, and we can then determine its use potential use-cases in relation to privacy. 
We will also explore the relevance of methods used in previous research projects, and how they could potentially be adapted for this new task. Additionally we will identify relevant privacy notions for information in this domain.

% maybe section on how anonymity is handled within ln later??