\chapter{Introduction}
\label{chap:introduction}

% quick background
Bitcoin is a electronic payment system relying on cryptography and a distributed ledger known as the blockchain \cite{nakamoto2008bitcoin}. 
The blockchain is used to record all transactions done in the system, and is maintained by every node in the network. New transactions is added to the blockchain by spending computational power; making the transactions practically irreversible as the blockchain grows and more computational power is used, and thus removes the need for trusted third parties to verify transactions. Such a decentralized payment solution does however not scale very well. It can maximally handle about 7 transaction per second \todo{add ref}, compared to visa and other centralized payment systems which can handle ? \todo{ref and numb}.
A proposed solution to increase the capacity and help scale Bitcoin is payment channel networks. Such a network would allow users to do  transactions without them being published in the Bitcoin network and therefore not included in the blockchain, which is why they are known as off-chain transactions. This allows for more transactions being done without increasing the capacity of the Bitcoin system itself. It can be thought of as a layer on top of Bitcoin since it require the Bitcoin system and builds on top of it. As the name implies it consist of networked payment channels. A payment channel is a one to one channel allowing the two participants to exchange founds between each other. A network is created by having multiple channels with different users and allowing for transfers to be done across multiple channels-e.g., Alice can pay Bob without having a channel directly connecting them, but using intermediary channels. All transactions done inside this network does not impact the Bitcoin network, however the opening and closing of channels does.
\\ 

%problem description
Because the blockchain is a public authoritative record of transactions done within the Bitcoin system, it has been used in research projects focusing on anonymity and privacy of users using the system \cite{reid2013analysis} \cite{meiklejohn2013fistful}. As the data in the blockchain provides the whole picture of how founds have moved, the focus of earlier research has been to analyze the data on the blockchain group this activity to the relevant users. This is done by grouping and contextualizing information, and successfully doing so will provide one with a record of all user activity. 
The Lightning Network (LN) as described in the paper by Poon and Dryja \cite{poon2015bitcoin} is a payment channel network currently used with Bitcoin \todo{ref?}. The use of this second layer network allowing transactions to be done off-chain makes this type of blockchain analysis more difficult. The lightning network is built on top of Bitcoin there is a need for on-chain transactions which will be recorded on the blockchain, and the data contained there will always be valid, but founds can move inside this second network which will not be observable on the blockchain in the same degree as previously. Jordi Herrera-Joancomartí and Cristina Pérez-Solà \cite{herrera2015research} state that the methods used for blockchain analysis in earlier research is no longer effective if payment channel networks would be widely used. If the Lightning Network (LN) is widely used, the majority of transactions would be done off-chain and only on-chain transactions would be the ones required for the payment channel to operate. Giulio Malavolta et al. \cite{malavolta2017concurrency} identifies some possible privacy implications of the on-chain payment network transactions, so there might be possibilities of using blockchain data related to these networks to compromise user privacy. Now that there are functioning payment channel networks such as the Lightning network we can explore what information will be available from the on-chain transactions of the LN, and not just in a theoretical manner but the impact of real user behaviour and large data sets. This paper will explore the possibility of blockchain analysis in relation to the lightning network, and to what degree it can impact privacy of the users. Doing this while the technology is still in a developmental stage and not yet widely used can be important as resolving potential privacy risks early is desirable.
\\

This project will explore the implications of LN from the blockchain perspective.
We will attempt to answer our research question which is: what information from the LN is available in the Bitcoin system and how does it impact the privacy of the users?
Our main goal which will allow us to answer this is to create a method for identifying transactions related to the lightning network in the blockchain, and discover what information doing this will provide us.
We will also explore the relevance of methods used in previous research projects and how they could potentially be adapted for this new task. Additionally we will identify relevant privacy notions for information in this domain.

% maybe section on how anonymity is handled within ln later??