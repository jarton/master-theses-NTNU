\chapter{Conclusion}
\label{chap:conclusion}

In this project we have explored the impact of the LN from the blockchain analysis perspective. As stated previously: the LN is layered on top of Bitcoin, meaning there is no real change in the Bitcoin system. Old methods of and discoveries related to blockchain analysis, not affected by user behaviour, is thus still valid. Taking the LN into consideration allows us to provide some context to the blockchain and the transactions within. In addition to seeing the blockchain as the ordering and consensus mechanism for the Bitcoin network, we must also see it as the mechanism for facilitating the payment channels crating the LN. The LN allows for transfers of Bitcoin, using Bitcoin transactions, and routed across multiple payment channels, but without being recorded in the blockchain. As these transfers is beyond the reach of blockchain analysis methods, and we have no method for inferring them from the LN on-chain transactions, the goal of previous projects to reveal user activity, can no longer be applied in our scenario. Instead we are limited to what information the on-chain transactions contain, which is payment channel information, so instead of revealing activity within the LN, we are limited to general information and structure of the LN. It is theretofore clear, from the blockchain analysis perspective, that doing transfers within the LN provides much better privacy than doing all transfers on-chain.
\\

We have in this project described methods for identifying on-chain LN transactions, and explored their effectiveness by comparing results with the LN. While unilaterally closed channels enable us to identify on-chain LN transactions, the lack of uniqueness in transactions resulting from other channels makes identification unreliable. We have also shown how previous heuristics used for linking information can be adapted to linking channels. While they allow for some linking to be done, comparing our results to the LN generated data shows how small percent of connections we can actually link. Our linking was however limited to using data from the channel graphs, and did not take into account any other transactions in the blockchain. If this was done, linking all keys, possibly allowing for users within channel graphs to be differentiated, one could recreate parts of the structure of the LN using this. Such a results, and our results to a degree can reveal how users connect within the LN, which shows us how transfers can be done, but without revealing how they are actually done.
But perhaps the fact that users choose to crate channels with some users instead of others, is something that can be used.
It is apparent that if one wants information about the LN, it is in most cases better to simply get this directly from the LN, instead of finding it on the blockchain. But as we previously have pointed out: all LN channels needs to be anchored on the blockchain, which itself will contain verified historical data, so it can be a important source of information when exploring the LN. Additionally, blockchain analysis can be used in conjunction with other methods of analysis related to the LN, which might have need for the information which can be found on the blockchain.

\section{Future Work}
\label{sec:future}

As a emerging technology the LN and Bitcoin system is being changed and improved regularly, and because the work presented here is for the current state of these systems, any future work using elements from this project, will have to change methods and assumptions. A example is founding transactions: currently founding transactions can be used to found a single channel, but there is nothing in the way of having a single founding transaction for many channels \cite{multi_channel_founding}. This would provide additional possibilities for linking. It could also be used for identification of channels-e.g., one channel might be identified because of a timelocked output, and by finding ouputs of other channels in the founding transaction we can also identify them.
\\

One of the proposed improvements that will impact this type of work the most is channel factories \cite{burchert2017scalable}.
The proposition is to have a additional layer between the blockchain and the payment channel network, with the purpose of creating the one-to-one payment channels which makes up the payment channel network. The channel factories on this layer can manage channels, meaning payment channels no longer requires any on-chain transactions to manage. The channel factories itself requires on-chain transactions, but with such a additional layer of abstraction, allowing for many dynamic channels per on-chain transactions, the use and methods of blockchain analysis must be reassessed.
\\