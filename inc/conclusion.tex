\chapter{Summary}
\label{chap:conclusion}

\section{Conclusion}
In this project we have explored the LN from the blockchain perspective.
As the LN does not change the Bitcoin system in any way, the older analysis methods are still valid if we don't consider the LN. Taking the LN into consideration means seeing transactions in the context of the LN in addition to the context of the blockchain. As we have discovered when attempting to locate on-chain LN transactions: one can never be perfectly sure that a transaction is LN related or not, if we only use the informatin available in the blockcahin. The different characteristics used to identify potential on-chain transactions have different level of uniqueness, but will limit the amount we can identify. Uniqueness makes it very unlikely that the transaction is not LN related, but as stated previously: users are free to create whatever scripts and transaction chains they wish. 
The identified channels does also contain limited amount of information allowing us to infer things about the LN, especially in terms of off-chain transaction. As the on-chain transactions facilities the channels in the LN, most of the information we get are related to these and its users.
While this information might allow for us to determine some of the structure of the LN, in terms of which channels are connected, it does not in our opinion help in determining anything about major regarding off-chain transaction flows. We can therefore conclude, from the blockchain analysis perspective, that doing transfers within the LN provides much better privacy than doing transfers on-chain.
\\


% identification
Our results also show some possibilities. In our second interval comparing the channels found with our two methods of identification, and channels collected trough the LN, we see that possibly more than 75\% of all 2of2 multisig transactions is in fact LN channels. This was due to the possible limits of our LN node to collect all channels, as other nodes will remove inactive channels, and they will therefore not be propagated. This might become apparent if one where to use longer collection intervals, or run the collection node longer before starting to collect, allowing the node to receive many channels that will eventually be pruned by other nodes.
The timelocked identification provides a good middle ground between precision and how many channels it will be able to find, with the results from the second interval indicating that around 25\% of channels is unilaterally closed. As shown in our correlation between {\tt P2WSH} outputs and number of channels on the LN, the {\tt P2WSH} outputs can be used as an indicator of the number of channels, at least providing a upper limit on active channels at any point in time. It is important to note however that these things are very reliant on the current use cases of the things used for identification and correlation. For example, if 2of2 multisig is used commonly for other porposes than LN channels, identification based on it will become extremely unreliable. 
\\


% linking
We should take note that much of established principles for linking information on the blockchain is still relevant for this case. Using new keys for each transaction providing the users with pseudoanonymity is very much still important for providing privacy in channels graphs. In addition to enabling us to link channels containing the same keys, it will also enable linking of keys within the channel graph itself, as keys are related to themselves. We discussed previously how differentiating users within a graph would enable us to determine which users provide link between channels, allowing us to recreate parts of the structure of the LN more accurately. The best way of doing this would be to link all the keys, and then the entire transaction graph should also be used.
Our linking results show that the information available and our heuristics are very limited. When comparing the channel network created by using information from the LN, with the one created using channel graphs found in the blockchain in the same interval, we can see a very clear difference in the number of edges. The network using LN information has a average degree over a hundred times larger than the channel network from the blockchain. The effectiveness of our heuristics is dependent on user behaviour, and how LN implementations operates in regard to transaction structure. For example, using a single transaction to spend many timelocked outputs is simple and convenient, but allows for linking using our third heuristic.
\\


It is apparent that if one wants information about the LN, it is in most cases better to simply get this directly from the LN, instead of finding it on the blockchain.
But as we previously have pointed out: all LN channels needs to be anchored on the blockchain, which contains verified historical data, so it will be a important source of information with regards to the LN. 
Additionally, blockchain analysis can be used in conjunction with other methods of analysis on the LN, providing a supplement. 

\section{Future Work}
\label{sec:future}

As a emerging technology the LN and Bitcoin system is being changed and improved regularly, and because the work presented here is for the current state of these systems, any future work using elements from this project, will have to change methods and assumptions.
A example is founding transactions: currently founding transactions can be used to found a single channel, but there is nothing in the way of having a single founding transaction for many channels \cite{multi_channel_founding}.
This would provide additional possibilities for linking.
It could also be used for identification of channels-e.g., one channel might be identified because of a timelocked output, and by finding ouputs of other channels in the founding transaction we can also identify them.
\\

Using HTLC transactions to identify on-chain LN transactions are something that could be explored further. In \cref{sec:fullrun} we presented our result of checking their frequency, and overlap with already found channel graph. Using such a identification method would have given us 62 more channels for the mainnet and 143 for the testnet. Nevertheless, the extra transactions included in a channel graph can provide more opportunities for linking. The scrips for HTLC transactions should also be explored further, to determine the value of the data within them.
The preimage {\tt R} is unique for one off-chain transaction, making it a possible element to use for getting information about off-chain transactions.
\\

One of the proposed improvements that will impact this type of work the most is channel factories \cite{burchert2017scalable}.
The proposition is to have a additional layer between the blockchain and the payment channel network, with the purpose of creating the one-to-one payment channels which makes up the payment channel network.
The channel factories on this layer can manage channels, meaning payment channels no longer requires any on-chain transactions to manage.
The channel factories itself requires on-chain transactions, but with such a additional layer of abstraction, allowing for many dynamic channels per on-chain transactions, the use and methods of blockchain analysis must be reassessed.
\\
