\chapter*{Abstract}

The blockchain is one of the main mechanisms enabling Bitcoin to be a decentralized electronic payment system. It provides the system with a shared transaction history, which can be verified by every participant. With increased use, the limited possibility for scaling to handle more transactions has become apparent. Payment channel networks are one proposed solution to this problem. They allows for more transactions to be done by moving some transactions to a separate network. The Lightning Network is one such network, which uses Bitcoin and the blockchain to operate. While the transactions in this network will not be included in the blockchain, there will be data there related to the network. This is because it needs the blockchain to manage the payment channels which the network consists of. 

In this project we have explored the blockchain with the goal of identifying transactions related to the Lightning Network, and by doing so, determine what information about it is available in the blockchain. We have created different methods for identifying these transactions. The methods use different transaction characteristics differing in uniqueness, making some methods more precise, but having fewer result and vice versa. We created software implementing the methods, which 
were used to parse the blockchain. The effectiveness of these methods have been quantified by comparing the data we found when parsing the blockchain, to data we collected directly from the Lightning Network. The results shows that the methods are viable for identifying a subset of transactions, and that precision can be sacrificed for finding more.
By identifying these transactions we were able to determine what information about the Lightning Network we can see from the blockchain perspective, and also some aspects were we are limited. 

We have also adapted heuristics from previous work doing blockchain analysis to our scenario.
These were used to link related information we had found when parsing the blockchain, which enabled us to create network graphs showing the relations between the Lightning Network channels identified on the blockchain. While the relations in this network graph were limited, compared to the actual relations found within the lightning network, they show how the blockchain can be used to infer non-explicit information about the lightning network. We have also identified several methods for potentially inferring or locating more information using what is available in the blockchain.

%The Bitcoin blockchian is the public shared ledger defining the shared transaction history for the Bitcoin system. But with the creation of payment channel networks such as the lightning network, the blockchain will also be used to facilitate transactions in a separate network. 
%While the transactions in this network will not impact the blockchain, the channels which this network consists of, does requires the blockchian to operate. These transactions are used to created and close channels, in addition they handle disagreements in the lightning network.
%By exploring the blockchain, we have created methods for identifying the transactions related to the lightning network. The methods use different transaction characteristics differing in uniqueness, making some methods more precise but having fewer result and vice versa.
%The effectiveness of these methods have been quantified by comparing it to data directly from the lightning network. The result shows that the methods are viable for finding a subset of transactions, and that precision can be sacrificed for finding even more.
%By identifying these transactions we have determined what information about the LN we can see from the blockchain perspective, and also where aspects that is limited. We have also adapted heuristics from previous work to these transactions, which we used to link related information. The linked information enabled us to create networks showing relations between the channels in the lightning network. 
%While being limited compared to the actual relations found within the lightning network, they show how the blockchain can be used to infer information about the lightning network. The work also identifies several potential methods for finding more information using what is available on the blockchain.
%
\hypersetup{pageanchor=false}